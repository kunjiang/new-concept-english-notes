\chapter{国际音标}


%%%%%%%%%%%%%%%%%%%%%%%%%%%%%%%%%%%%%%%%%%%%%%%%%%%
%  内容来源于网络, 重新修改
%%%%%%%%%%%%%%%%%%%%%%%%%%%%%%%%%%%%%%%%%%%%%%%%%%%
%
%
% 英语国际音标有 48 个. 其中元音 20 个, 辅音 28 个.

% % 输入参考: https://www.doc88.com/p-2426179336421.html?r=1
% % 查看文档: 在命令行中输入 texdoc tipa


% \begin{center}
%   \begin{tabular}{|c|c|c|c|c|c|c|c|c|}
%     \multicolumn{9}{c}{元音 20 个} \\
%     \hline
%     单元音 & 长元音 & 
%       \jkipa{{i\textlengthmark}} & \jkipa{\textrevepsilon\textlengthmark} & \jkipa{u\textlengthmark} &
%       \jkipa{\textopeno\textlengthmark} & \jkipa{a\textlengthmark} & & \\
%     \hline
%     单元音 & 短元音 &
%       \jkipa{\i} & \jkipa{\textschwa} & \jkipa{\textupsilon} & \jkipa{\textturnscripta} &
%       \jkipa{\textturnv} & \jkipa{e} & \jkipa{\ae} \\
%     \hline
%     双元音 & 合口双元音 & 
%       \jkipa{{e\i}} & \jkipa{{a\i}} & \jkipa{{\textopeno\i}} & \jkipa{{\textschwa\textupsilon}} & \jkipa{{a\textupsilon}} & & \\ 
%     \hline
%     双元音 & 集中双元音 &
%       \jkipa{{\i\textschwa}} & \jkipa{{e\textschwa}} & \jkipa{{\textupsilon\textschwa}} & & & & \\
%     \hline
%   \end{tabular}
% \end{center}


% \begin{center}
%   \begin{tabular}{|c|c|c|c|c|c|c|c|c|c|c|}
%     \multicolumn{11}{c}{辅音 28 个} \\
%     \hline
%     清辅音 & \jkipa{p} & \jkipa{t} & \jkipa{k} & \jkipa{f} & \jkipa{\texttheta} & \jkipa{s} &
%           \jkipa{\texttslig} & \jkipa{\textesh} & \jkipa{\textteshlig} & \jkipa{{tr}} \\
%     \hline
%     浊辅音 & \jkipa{b} & \jkipa{d} & \jkipa{\textscriptg} & \jkipa{v} & \jkipa{\dh} & \jkipa{z} &
%           \jkipa{\textdzlig} & \jkipa{\textyogh} & \jkipa{\textdyoghlig} &  \jkipa{{dr}} \\
%     \hline
%     鼻音 & \jkipa{m} & \jkipa{n} & \jkipa{\ng} & & & & & & & \\
%     \hline
%     微浊辅音 & \jkipa{h} & \jkipa{r} & & & & & & & & \\
%     \hline
%     舌侧音 & \jkipa{l} & & & & & & & & & \\
%     \hline
%     半元音 & \jkipa{w} & \jkipa{j}  & & & & & & & & \\
%     \hline
%   \end{tabular}
% \end{center}

% \noindent 注意: 来源于网络, 待深入整理. % 待用牛津或朗文词典中描述的音标进行整理


%
% TIPA 包中, 长音两点使用符号 \textlengthmark 来表示
% 输入反过来的内容使用 turn 和 rev


这里使用牛津高阶第四版的参考.

\section*{音标例释 (Key Phonetic Symbols)}

音标分为元音和辅音 (Consonants), 元音又包含单元音 (Vowels) 与双元音 (Diphthongs).
除此之外, 音标分为英式音标 (Jones) 与美式音标 (K.K.)\footnote{K.K. 音标可能已退出舞台, 朗文高阶第五版未出现 K.K. 音标, 而牛津第八版也未出现 K.K. 音标. 情怀一下吧.}.

\begin{center}
  \begin{tabular}[t]{|c|c|l|}
    \hline
    \multicolumn{2}{|c|}{IPA 国际音标} & \\
    \cline{1-2}
    Jones & K.K. & \raisebox{1.6ex}[0pt]{示例 Example} \\
    \hline
    i\textlengthmark  & i               & \textbf{see} \jkipa{si\textlengthmark; si} \\
    \hline
    \textsci          & \textsci        & \textbf{sit} \jkipa{s\textsci; s\textsci} \\
    \hline
    e                 & \textepsilon    & \textbf{ten} \jkipa{ten; t\textepsilon n} \\
    \hline
    \ae               & \ae             & \textbf{hat} \jkipa{h\ae t; h\ae t} \\
    \hline
    \textscripta\textlengthmark & \textscripta & \textbf{palm} \jkipa{p\textscripta\textlengthmark m; p\textscripta m} \\
    \hline
                      & \ae             & \textbf{ask} \jkipa{\textscripta\textlengthmark sk; \ae sk} \\
    \hline
    % \textturnscripta  & \textscripta    & \textbf{watch} \jkipa{w\textturnscripta\textteshlig; w\textscripta\textteshlig} \\
    \textturnscripta  & \textscripta    & \textbf{watch} \jkipa{w\textturnscripta t\textesh; w\textscripta t\textesh} \\
    \hline
                      & \textopeno      & \textbf{long} \jkipa{l\textturnscripta\ng; l\textopeno\ng} \\
    \hline
    \textopeno\textlengthmark & \textopeno & \textbf{saw} \jkipa{s\textopeno\textlengthmark; s\textopeno} \\
    \hline
    \textupsilon      & \textscu        & \textbf{put} \jkipa{p\textupsilon t; p\textscu t} \\
    \hline
    u\textlengthmark  & u               & \textbf{too} \jkipa{tu\textlengthmark; tu} \\
    \hline
    \textturnv        & \textturnv      & \textbf{cup} \jkipa{k\textturnv p; k\textturnv p} \\
    \hline
  \end{tabular}
%
  \begin{tabular}[t]{|c|c|l|}
    \hline
    \multicolumn{2}{|c|}{IPA 国际音标} & \\
    \cline{1-2}
    Jones & K.K. & \raisebox{1.6ex}[0pt]{示例 Example} \\
    \hline
    \textrevepsilon\textlengthmark & \textrhookrevepsilon & \textbf{fur} \jkipa{f\textrevepsilon\textlengthmark(r); f\textrhookrevepsilon} \\
    \hline
    \textschwa        & \textschwa      & \textbf{ago} \jkipa{\textschwa\textprimstress\textscriptg\textschwa\textupsilon; \textschwa\textprimstress go} \\
    \hline
                      & \textrhookrevepsilon & \textbf{never} \jkipa{\textprimstress nev\textschwa(r); \textprimstress n\textepsilon v\textrhookrevepsilon} \\   
    \hline
    % e\textsci         & e               & \textbf{page} \jkipa{pe\textsci\textdyoghlig; pe\textdyoghlig} \\ 
    e\textsci         & e               & \textbf{page} \jkipa{pe\textsci d\textyogh; ped\textyogh} \\ 
    \hline
    \textschwa\textupsilon  & o & \textbf{home} \jkipa{h\textschwa\textupsilon m; hom} \\
    \hline
    a\textsci        & a\textsci       & \textbf{five} \jkipa{fa\textsci v; fa\textsci v} \\
    \hline
    a\textupsilon    & a\textscu        & \textbf{now} \jkipa{na\textupsilon; na\textscu} \\
    \hline
    \textopeno\textsci & \textopeno\textsci & \textbf{join} \jkipa{d\textyogh\textopeno\textsci; d\textyogh\textopeno\textsci} \\
    \hline
    \textsci\textschwa & \textsci r     & \textbf{near} \jkipa{n\textsci\textschwa(r); n\textsci r} \\
    \hline
    e\textschwa       & \textepsilon r  & \textbf{hair} \jkipa{he\textschwa(r); h\textepsilon r}\\
    \hline
    \textupsilon\textschwa & \textscu r & \textbf{tour} \jkipa{t\textupsilon\textschwa(r); t\textscu r} \\
    \hline
  \end{tabular}
\end{center}

% 将连在一起的音标 teshlig 与 dyoghlig 分开, 看起来舒服点. 